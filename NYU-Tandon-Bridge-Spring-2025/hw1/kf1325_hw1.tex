\documentclass[10pt]{article}
\usepackage[utf8]{inputenc}
\usepackage[T1]{fontenc}
\usepackage{amsmath}
\usepackage{amsfonts}
\usepackage{amssymb}
\usepackage[version=4]{mhchem}
\usepackage{stmaryrd}
\usepackage{array,mathtools}
\newcommand*{\carry}[1][1]{\overset{#1}}
\newcolumntype{B}[1]{r*{#1}{@{\,}r}}

\begin{document}
\section*{Homework 1}
Student: Kevin Fang (kf1325)

\section*{Question 1}
\section*{A.}
\subsection*{1.}

$$
\begin{aligned}
10011011_{2} & =1 \times 2^{7}+0 \times 2^{6}+0 \times 2^{5}+1 \times 2^{4}+1 \times 2^{3}+0 \times 2^{2}+1 \times 2^{1}+1 \times 2^{0} \\
& =128+0+0+16+8+0+2+1 \\
& =155
\end{aligned}
$$

\subsection*{2.}

$$
\begin{aligned}
456_{7} & =4 \times 7^{2}+5 \times 7^{1}+6 \times 7^{0} \\
& =196+35+6 \\
& =237
\end{aligned}
$$

\subsection*{3.}

$$
\begin{aligned}
38 A_{16} & =3 \times 16^{2}+8 \times 16^{1}+10 \times 16^{0} \\
& =768+128+10 \\
& =906
\end{aligned}
$$

\subsection*{4.}

$$
\begin{aligned}
2214_{5} & =2 \times 5^{3}+2 \times 5^{2}+1 \times 5^{1}+4 \times 5^{0} \\
& =250+50+5+4 \\
& =309
\end{aligned}
$$

\pagebreak
\section*{B.}
\subsection*{1.}

$$
\begin{aligned}
69 \div 2 &= 34 &\quad \bmod 1 \\
34 \div 2 &= 17 &\quad \bmod 0 \\
17 \div 2 &= 8  &\quad \bmod 1 \\
8 \div 2  &= 4  &\quad \bmod 0 \\
4 \div 2  &= 2  &\quad \bmod 0 \\
2 \div 2  &= 1  &\quad \bmod 0 \\
1 \div 2  &= 0  &\quad \bmod 1 \\
69_{10}   &= 1000101_{2}
\end{aligned}
$$

\subsection*{2.}

$$
\begin{array}{rlr}
485 \div 2 & =242 &\quad \bmod 1 \\
242 \div 2 & =121 &\quad \bmod 0 \\
121 \div 2 & =60 &\quad \bmod 1 \\
60 \div 2 & =30 &\quad \bmod 0 \\
30 \div 2 & =15 &\quad \bmod 0 \\
15 \div 2 & =7 &\quad \bmod 1 \\
7 \div 2 & =3 &\quad \bmod 1 \\
3 \div 2 & =1 &\quad \bmod 1 \\
1 \div 2 & =0 &\quad \bmod 1 \\
& \\
485_{10} & =111100101_{2}
\end{array}
$$

\subsection*{3.}

\begin{flushleft}
We translate each hexadecimal digit into a 4-bit binary equivalent.

$$
\begin{aligned}
6_{16} &= 0110_{2} \\
D_{16} &= 13_{10} &= 1101_{2} \\
1_{16} &= 0001_{2} \\
A_{16} &= 10_{10} &= 1010_{2}
\end{aligned}
$$

Thus, combining the binary values, We get,

$$
6 D 1 A_{16}=0110110100011010_{2}
$$
\end{flushleft}

\section*{C.}
\subsection*{1.}

$$
1101011_{2}=01101011_{2}
$$

We translate each hexadecimal digit into a 4-bit binary equivalent.

$$
\begin{aligned}
& 0110_{2}=6_{16} \\
& 1011_{2}=B_{16}
\end{aligned}
$$

Thus, combining the binary values, We get,

$$
1101011_{2}=6 B_{16}
$$

\subsection*{2.}

$$
\begin{aligned}
895 \div 16 & =55 & & \bmod 15 \\
55 \div 16 & =3 & & \bmod 7 \\
3 \div 16 & =0 & & \bmod 3 \\
15_{10} & =F_{16} & & \\
895_{10} & =37 F_{16} & &
\end{aligned}
$$

\pagebreak
\section*{Question 2}
\subsection*{1.}


$$
\begin{array}{@{}B2}
\carry 0\carry 7\carry 5\carry 6 6_{8} \\
+4515_{8} \\\hline 
14303_{8} \\
\end{array}
$$

$$
7566_{8}+4515_{8}=14303_{8}
$$

\subsection*{2.}

$$
\begin{array}{@{}B2}
10\carry 1\carry 1\carry 0\carry 0\carry 11_{2} \\
+00001101_{2} \\\hline 
11000000_{2} \\
\end{array}
$$

$$
10110011_{2}+1101_{2}=11000000_{2}
$$

\subsection*{3.}

$$
\begin{array}{@{}B2}
\carry 7& \carry A&66_{16} \\
+4&5&C5_{16} \\\hline 
C&0&2B_{16} \\
\end{array}
$$

$$
7 A 66_{16}+45 C 5_{16}=C 02 B_{16}
$$

\subsection*{4.}

$$
\begin{array}{@{}B2}
\carry [2] 3 \carry [4] 0 \carry 22_{5} \\
-2433_{5} \\\hline 
34_{5}
\end{array}
$$

$$
3022_{5}-2433_{5}=34_{5}
$$

\pagebreak
\section*{Question 3}
\section*{A.}
\subsection*{1.}

$$
\begin{array}{rlr}
124 \div 2 & =62 & \bmod 0 \\
62 \div 2 & =31 & \bmod 0 \\
31 \div 2 & =15 & \bmod 1 \\
15 \div 2 & =7 & \bmod 1 \\
7 \div 2 & =3 & \bmod 1 \\
3 \div 2 & =1 & \bmod 1 \\
1 \div 2 & =0 & \bmod 1
\end{array}
$$

$$
124_{10}=01111100 \textsubscript{ 8 bit 2's comp}
$$

\subsection*{2.}

$$
\begin{aligned}
124 \div 2 & =62 & \bmod 0 \\
62 \div 2 & =31 & \bmod 0 \\
31 \div 2 & =15 & \bmod 1 \\
15 \div 2 & =7 & \bmod 1 \\
7 \div 2 & =3 & \bmod 1 \\
3 \div 2 & =1 & \bmod 1 \\
1 \div 2 & =0 & \bmod 1 \\
\end{aligned}
$$

Hence, the absolute binary representation of 124 is as followed:

$$
\begin{aligned}
124_{10} & =01111100_{2}
\end{aligned}
$$

The inverted bits are as followed:

$$
10000011
$$

Add 1,

$$
10000011+00000001=10000100
$$

Thus,

$$
-124_{10}=10000100 \textsubscript{ 8 bit 2's comp}
$$

\subsection*{3.}

$$
\begin{aligned}
109 \div 2 &= 54 & \bmod 1 \\
54 \div 2 &= 27 & \bmod 0 \\
27 \div 2 &= 13 & \bmod 1 \\
13 \div 2 &= 6 & \bmod 1 \\
6 \div 2 &= 3 & \bmod 0 \\
3 \div 2 &= 1 & \bmod 1 \\
1 \div 2 &= 0 & \bmod 1 \\
\end{aligned}
$$

$$
109_{10}=01101101 \textsubscript{ 8 bit 2's comp}
$$

\subsection*{4.}

$$
\begin{aligned}
79 \div 2 & =39 & \bmod 1 \\
39 \div 2 & =19 & \bmod 1 \\
19 \div 2 & =9 & \bmod 1 \\
9 \div 2 & =4 & \bmod 1 \\
4 \div 2 & =2 & \bmod 0 \\
2 \div 2 & =1 & \bmod 0 \\
1 \div 2 & =0 & \bmod 1 \\
\end{aligned}
$$

Hence, the absolute binary representation of 79 is as followed:

$$
\begin{aligned}
79_{10} & =01001111_{2}
\end{aligned}
$$

The inverted bits are as followed:

$$
10110000
$$

Add 1,

$$
10110000+00000001=10110001
$$

Then,

$$
-79_{10}=10110001 \textsubscript{ 8 bit 2's comp}
$$

\section*{B.}
\subsection*{1.}

$$
\begin{aligned}
00011110 \textsubscript{ 8 bit 2's comp} & =
-0 \times 2^{7}
+0 \times 2^{6}
+0 \times 2^{5}
+1 \times 2^{4}
+1 \times 2^{3}
+1 \times 2^{2}
+1 \times 2^{1}
+0 \times 2^{0} \\
& =-0+0+0+16+8+4+2+0 \\
& =30
\end{aligned}
$$

\subsection*{2.}


$$
\begin{aligned}
11100110 \textsubscript{ 8 bit 2's comp} & =
-1 \times 2^{7}
+1 \times 2^{6}
+1 \times 2^{5}
+0 \times 2^{4}
+0 \times 2^{3}
+1 \times 2^{2}
+1 \times 2^{1}
+0 \times 2^{0} \\
& =-128+64+32+0+0+4+2+0 \\
& =-26
\end{aligned}
$$

\subsection*{3.}

$$
\begin{aligned}
00101101 \textsubscript{ 8 bit 2's comp} & =
-0 \times 2^{7}
+0 \times 2^{6}
+1 \times 2^{5}
+0 \times 2^{4}
+1 \times 2^{3}
+1 \times 2^{2}
+0 \times 2^{1}
+1 \times 2^{0} \\
& =-0+0+32+0+8+4+0+1 \\
& =45
\end{aligned}
$$

\subsection*{4.}

$$
\begin{aligned}
10011110 \textsubscript{ 8 bit 2's comp} & =
-1 \times 2^{7}
+0 \times 2^{6}
+0 \times 2^{5}
+1 \times 2^{4}
+1 \times 2^{3}
+1 \times 2^{2}
+1 \times 2^{1}
+0 \times 2^{0} \\
& =-128+0+0+16+8+4+2+0 \\
& =-98
\end{aligned}
$$

\pagebreak
\section*{Question 4}
\section*{1. Exercise 1.2.4}
\section*{section b}
\begin{center}
\begin{tabular}{|c|c|c|c|}
\hline
$p$ & $q$ & $p \vee q$ & $\neg(p \vee q)$ \\
\hline
T & T & T & F \\
\hline
T & F & T & F \\
\hline
F & T & T & F \\
\hline
F & F & F & T \\
\hline
\end{tabular}
\end{center}

\section*{section c}
\begin{center}
\begin{tabular}{|c|c|c|c|c|}
\hline
$p$ & $q$ & $r$ & $p \wedge \neg q$ & $r \vee(p \wedge \neg q)$ \\
\hline
T & T & T & F & T \\
\hline
T & T & F & F & F \\
\hline
T & F & T & T & T \\
\hline
T & F & F & T & T \\
\hline
F & T & T & F & T \\
\hline
F & T & F & F & F \\
\hline
F & F & T & F & T \\
\hline
F & F & F & F & F \\
\hline
\end{tabular}
\end{center}

\section*{2. Exercise 1.3.4}
\section*{section b}
\begin{center}
\begin{tabular}{|c|c|c|c|c|}
\hline
$p$ & $q$ & $p \rightarrow q$ & $q \rightarrow p$ & $(p \rightarrow q) \rightarrow(q \rightarrow p)$ \\
\hline
T & T & T & T & T \\
\hline
T & F & F & T & T \\
\hline
F & T & T & F & F \\
\hline
F & F & T & T & T \\
\hline
\end{tabular}
\end{center}

\section*{section d}
\begin{center}
\begin{tabular}{|c|c|c|c|c|}
\hline
$p$ & $q$ & $p \leftrightarrow q$ & $p \leftrightarrow \neg q$ & $(p \leftrightarrow q) \oplus(p \leftrightarrow \neg q)$ \\
\hline
T & T & T & F & T \\
\hline
T & F & F & T & T \\
\hline
F & T & F & T & T \\
\hline
F & F & T & F & T \\
\hline
\end{tabular}
\end{center}

\pagebreak
\section*{Question 5}
\section*{1. Exercise 1.2.7}
\section*{section b}

$$
(B \wedge D) \vee(B \wedge M) \vee(D \wedge M)
$$

\section*{section c}

$$
B \vee(D \wedge M)
$$

\section*{2. Exercise 1.3.7}
\section*{section b}

$$
(s \vee y) \rightarrow p
$$

\section*{section c}

$$
p \rightarrow y
$$

\section*{section d}

$$
p \leftrightarrow(s \wedge y)
$$

\section*{section e}

$$
p \rightarrow(s \vee y)
$$

\section*{3. Exercise 1.3.9}
\section*{section c}

$$
c \rightarrow p
$$

\section*{section d}

$$
c \rightarrow p
$$

\pagebreak
\section*{Question 6}
\section*{1. Exercise 1.3.6}
\section*{section b}
If Joe is eligible for the honors program, then he maintains a B average.

\section*{section c}
If Rajiv can go on the roller coaster, then he is at least four feet tall.

\section*{section d}
If Rajiv is at least four feet tall, then he can go on the roller coaster.

\section*{2. Exercise 1.3.10}
\section*{section c}
\begin{center}
\begin{tabular}{|c|c|c|c|c|c|}
\hline
$p$ & $q$ & $r$ & $p \vee r$ & $q \wedge r$ & $(p \vee r) \leftrightarrow(q \wedge r)$ \\
\hline
T & F & T & T & F & F \\
\hline
T & F & F & T & F & F \\
\hline
\end{tabular}
\end{center}

The truth table indicates that the truth value of the expression $(p \vee r) \leftrightarrow(q \wedge r)$ is false.

\section*{section d}
\begin{center}
\begin{tabular}{|c|c|c|c|c|c|}
\hline
$p$ & $q$ & $r$ & $p \wedge r$ & $q \wedge r$ & $(p \wedge r) \leftrightarrow(q \wedge r)$ \\
\hline
T & F & T & T & F & F \\
\hline
T & F & F & F & F & T \\
\hline
\end{tabular}
\end{center}

The truth table indicates that the truth value of the expression $(p \wedge r) \leftrightarrow(q \wedge r)$ is unknown.

\section*{section e}
\begin{center}
\begin{tabular}{|c|c|c|c|c|}
\hline
$p$ & $q$ & $r$ & $r \vee q$ & $p \rightarrow(r \vee q)$ \\
\hline
T & F & T & T & T \\
\hline
T & F & F & F & F \\
\hline
\end{tabular}
\end{center}

The truth table indicates that the truth value of the expression $p \rightarrow(r \vee q)$ is unknown.

\section*{section $\mathbf{f}$}
\begin{center}
\begin{tabular}{|c|c|c|c|c|}
\hline
$p$ & $q$ & $r$ & $p \wedge q$ & $(p \wedge q) \rightarrow r$ \\
\hline
T & F & T & F & T \\
\hline
T & F & F & F & T \\
\hline
\end{tabular}
\end{center}

The truth table indicates that the truth value of the expression $(p \wedge q) \rightarrow r$ is true.

\pagebreak
\section*{Question 7}
\section*{Exercise 1.4.5}
\section*{section b}

The logical expression of the first sentence is as follows:

$$
\neg j \rightarrow(l \vee \neg r)
$$

The logical expression of the second sentence is as follows:

$$
(r \wedge \neg l) \rightarrow j
$$

Proof:

\begin{center}
\begin{tabular}{|c|c|c|c|c|}
\hline
$j$ & $l$ & $r$ & $\neg j \rightarrow(l \vee \neg r)$ & $(r \wedge \neg l) \rightarrow j$ \\
\hline
T & T & T & T & T \\
\hline
T & T & F & T & T \\
\hline
T & F & T & T & T \\
\hline
T & F & F & T & T \\
\hline
F & T & T & T & T \\
\hline
F & T & F & T & T \\
\hline
F & F & T & F & F \\
\hline
F & F & F & T & T \\
\hline
\end{tabular}
\end{center}

The truth table indicates that the two expressions have the same truth value in all situations; therefore, they are logically equivalent.

\section*{section c}
The logical expression of the first sentence is as follows:

$$
j \rightarrow \neg l
$$

The logical expression of the second sentence is as follows:

$$
\neg j \rightarrow l
$$

Proof:

\begin{center}
\begin{tabular}{|c|c|c|c|}
\hline
$j$ & $l$ & $j \rightarrow \neg l$ & $\neg j \rightarrow l$ \\
\hline
T & T & F & T \\
\hline
T & F & T & T \\
\hline
F & T & T & T \\
\hline
F & F & T & F \\
\hline
\end{tabular}
\end{center}

The truth table indicates that the two expressions have different truth values; therefore, they are not logically equivalent.

\section*{section d}
The logical expression of the first sentence is as follows:

$$
(r \vee \neg l) \rightarrow j
$$

The logical expression of the second sentence is as follows:

$$
j \rightarrow(r \wedge \neg l)
$$

Proof:

\begin{center}
\begin{tabular}{|c|c|c|c|c|}
\hline
$j$ & $l$ & $r$ & $(r \vee \neg l) \rightarrow j$ & $j \rightarrow(r \wedge \neg l)$ \\
\hline
T & T & T & T & F \\
\hline
T & T & F & T & F \\
\hline
T & F & T & T & T \\
\hline
T & F & F & T & F \\
\hline
F & T & T & F & T \\
\hline
F & T & F & T & T \\
\hline
F & F & T & F & T \\
\hline
F & F & F & F & T \\
\hline
\end{tabular}
\end{center}

The truth table indicates that the two expressions have different truth values; therefore, they are not logically equivalent.

\pagebreak
\section*{Question 8}
\section*{1. Exercise 1.5.2}
\section*{section c}

$$
\begin{aligned}
(p \rightarrow q) \wedge(p \rightarrow r) & \equiv(\neg p \vee q) \wedge(p \rightarrow r) & \text { (Conditional identity) } \\
& \equiv(\neg p \vee q) \wedge(\neg p \vee r) & \text { (Conditional identity) } \\
& \equiv \neg p \wedge(q \vee r) & \text { (Distributive law) } \\
& \equiv p \rightarrow(q \vee r) & \text { (Conditional identity) }
\end{aligned}
$$

\section*{section f}

$$
\begin{aligned}
\neg(p \vee(\neg p \wedge q)) & \equiv \neg((p \vee \neg p) \wedge(p \vee q)) & \text {(Distributive law)}\\
& \equiv \neg(T \wedge(p \vee q)) & \text {(Complement law)}\\
& \equiv \neg(p \vee q)  & \text {(Identity law)}\\
& \equiv \neg p \wedge \neg q & \text {(De Morgan's law)} \\
\end{aligned}
$$

\section*{section i}

$$
\begin{aligned}
(p \wedge q) \rightarrow r & \equiv \neg(p \wedge q) \vee r & \text {(Conditional identity)}\\
& \equiv(\neg p \vee \neg q) \vee r                         & \text {(De Morgan's law)}\\
& \equiv r \vee(\neg p \vee \neg q)                         & \text {(Commutative law)}\\
& \equiv(r \vee \neg p) \vee \neg q                         & \text {(Associative law)}\\
& \equiv(\neg p \vee r) \vee \neg q                         & \text {(Commutative law)}\\
& \equiv(\neg p \vee \neg \neg r) \vee \neg q               & \text {(Double negation law)}\\
& \equiv \neg(p \wedge \neg r) \vee \neg q                  & \text {(De Morgan's law)}\\
& \equiv(p \wedge \neg r) \rightarrow \neg q                & \text {(Conditional identity)}\\
\end{aligned}
$$

\section*{2. Exercise 1.5.3}
\section*{section c}

$$
\begin{aligned}
\neg r \vee(\neg r \rightarrow p) & \equiv \neg r \vee(\neg \neg r \vee p) & \text {(Conditional identity)}\\
& \equiv \neg r \vee(r \vee p)                                             & \text {(Double negation law)}\\
& \equiv(\neg r \vee r) \vee p                                             & \text {(Associative law)}\\
& \equiv T \vee p                                                          & \text {(Complement law)}\\
& \equiv T                                                                   & \text {(Domination law)}\\
\end{aligned}
$$

\section*{section d}
$$
\begin{aligned}
\neg(p \rightarrow q) \rightarrow \neg q & \equiv \neg \neg(p \rightarrow q) \vee \neg q & \text {(Conditional identity)}\\
& \equiv(p \rightarrow q) \vee \neg q                                                    & \text {(Double negation law)}\\
& \equiv(\neg p \vee q) \vee \neg q                                                      & \text {(Conditional identity)}\\
& \equiv \neg p \vee(q \vee \neg q)                                                      & \text {(Associative law)}\\
& \equiv \neg p \vee T                                                                   & \text {(Complement law)}\\
& \equiv T                                                                               & \text {(Domination law)}\\
\end{aligned}
$$

\pagebreak
\section*{Question 9}
\section*{1. Exercise 1.6.3}
\section*{section c}

$$
\exists x\left(x=x^{2}\right)
$$

\section*{section d}

$$
\forall x\left(x \leq x^{2}+1\right)
$$

\section*{2. Exercise 1.7.4}
\section*{section b}
$$
\forall x(\neg S(x) \wedge W(x))
$$

\section*{section c}

$$
\forall x(S(x) \rightarrow \neg W(x))
$$

\section*{section d}

$$
\exists x(S(x) \wedge W(x))
$$

\pagebreak
\section*{Question 10}
\section*{1. Exercise 1.7.9}
\section*{section c}
True.\\
The expression indicates that there exists an $x$ such that either $x=c$ is false or $P(x)$ is true. The truth table indicates that if $x=a, x=c$ is false; thus, the expression is true.

\section*{section d}
True.\\
The truth table indicates that if $x=e$, then both $Q(x)$ and $R(x)$ are true so that $Q(x) \wedge R(x)$ is true. Therefore, there exists an $x$ such that $Q(x) \wedge R(x)$ is true.

\section*{section e}
True.\\
The truth table indicates that both $Q(a)$ and $P(d)$ is true. Therefore, $Q(a) \wedge P(d)$ is true.

\section*{section f}
True.\\
The truth table indicates that for all $x$, where $x \neq b, Q(x)$ is always true.

\section*{section g}
False.\\
The truth table indicates when $x=c$, both $P(x)$ and $R(x)$ are false, so that $P(x) \vee R(x)$ is false. Therefore, for all $x, P(x) \vee R(x)$ is false.

\section*{section h}
True.\\
According to the truth table, for all $x$, either $P(x)$ is true or $R(x)$ is false. If $P(x)$ is true, then $R(x) \rightarrow P(x)$ is true. If $R(x)$ is false, then $R(x) \rightarrow P(x)$ is also true. Therefore, for all $x, R(x) \rightarrow P(x)$ is always true.

\section*{section i}
True.\\
The truth table indicates that when $x=a$, $Q(x)$ is true and $R(x)$ is false so that $Q(x) \vee R(x)$ is true. Therefore, there exists an $x$ such that $Q(x) \vee R(x)$ is true.

\section*{2. Exercise 1.9.2}
\section*{section b}
True.\\
The truth table indicates that when $x=2, Q(x, y)$ is true for all $y$.\\
\section*{section c}
True.\\
The truth table indicates that when $y=1$, $P(x, y)$ is true for all $x$.\\
\section*{section d}
False.\\
The truth table indicates that $S(x, y)$ is always false. Therefore, there are no $x$ and a $y$ such that $S(x, y)$ is true.

\section*{section e}
False.\\
The truth table indicates that when $x=1$, there does not exist a $y$ such that $Q(x, y)$ is true. Therefore, the statement "for all $x$, there exists a $y$ such that $Q(x, y)$ is true" is false.

\section*{section f}
True.\\
The truth table indicates that for all $x$, there exists a $y$ such that $P(x, y)$ is true. When $x=1$, if $y=1$, then $P(x, y)$ is true. When $x=2$, if $y=1$, then $P(x, y)$ is true. When $x=3$, if $y=1$, then $P(x, y)$ is true.

\section*{section g}
False.\\
The truth table indicates that when $x=1$ and $y=2$, $P(x, y)$ is false. Therefore, the statement that "for all $x$ and $y, P(x, y)$ is true" is false.

\section*{section h}
True.\\
The truth table indicates that when $x=2$ and $y=1, Q(x, y)$ is true. Therefore, there exist an $x$ and a $y$ such that $Q(x, y)$ is true.

\section*{section i}
True.\\
The truth table indicates that for all $x$ and $y, S(x, y)$ is always false. Therefore, $\neg Q(x, y)$ is always true.

\pagebreak
\section*{Question 11}
\section*{1. Exercise 1.10.4}
\section*{section c}
$$
\exists x \exists y(x+y=x y)
$$

\section*{section d}

$$
\forall x \forall y\left(((x>0) \wedge(y>0)) \rightarrow\left(\frac{x}{y}>0\right)\right)
$$

\section*{section e}

$$
\forall x\left(((x>0) \wedge(x<1)) \rightarrow\left(\frac{1}{x}>1\right)\right)
$$

\section*{section f}

$$
\forall x \exists y(y<x)
$$

\section*{section g}

$$
\forall x \exists y\left((x \neq 0) \rightarrow\left(y=\frac{1}{x}\right)\right)
$$

\section*{2. Exercise 1.10.7}
\section*{section c}
$$
\exists x(N(x) \wedge D(x))
$$

\section*{section d}

$$
\forall y(D(y) \rightarrow P(\mathrm{Sam}, y))
$$

\section*{section e}

$$
\exists x \forall y(N(x) \wedge P(x, y))
$$

\section*{section f}

$$
\exists x \forall y((N(x) \wedge D(x)) \wedge((N(y) \wedge(y \neq x)) \rightarrow \neg D(y)))
$$

\section*{3. Exercise 1.10.10}
\section*{section c}

$$
\forall x \exists y(T(x, y) \wedge(y \neq \text { Math 101 }))
$$

\section*{section d}

$$
\exists x \forall y((y \neq \text { Math 101 }) \rightarrow T(x, y))
$$

\section*{section e}

$$
\forall x \exists y_{1} \exists y_{2}\left((x \neq \operatorname{Sam}) \rightarrow\left(T\left(x, y_{1}\right) \wedge T\left(x, y_{2}\right) \wedge\left(y_{1} \neq y_{2}\right)\right)\right)
$$

\section*{section f}
$\left.\exists y_{1} \exists y_{2} \forall y_{3}\left(T\left(\operatorname{Sam}, y_{1}\right) \wedge T\left(\operatorname{Sam}, y_{2}\right) \wedge\left(y_{1} \neq y_{2}\right) \wedge\left(y_{3} \neq y_{1}\right) \wedge\left(y_{3} \neq y_{2}\right)\right) \rightarrow \neg T\left(\operatorname{Sam}, y_{3}\right)\right)$

\pagebreak
\section*{Question 12}
\section*{1. Exercise 1.8.2}
\section*{section b}
Logical Expression:

$$
\forall x(D(x) \vee P(x))
$$

Negation:

$$
\neg \forall x(D(x) \vee P(x))
$$

Applying De Morgan's law:

$$
\exists x \neg(D(x) \vee P(x))
$$

Applying De Morgan's law:

$$
\exists x(\neg D(x) \wedge \neg P(x))
$$

English: There is a patient who received neither the medication nor the placebo.

\section*{section c}
Logical Expression:

$$
\exists x(D(x) \wedge M(x))
$$

Negation:

$$
\neg \exists x(D(x) \wedge M(x))
$$

Applying De Morgan's law:

$$
\forall x \neg(D(x) \wedge M(x))
$$

Applying De Morgan's law:

$$
\forall x(\neg D(x) \vee \neg M(x))
$$

English: Not every patients were given the medication nor did they have migraines or both.

\section*{section d}
Logical Expression:

$$
\forall x(P(x) \rightarrow M(x))
$$

Negation:

$$
\neg \forall x(P(x) \rightarrow M(x))
$$

Applying De Morgan's law:

$$
\exists x \neg(P(x) \rightarrow M(x))
$$

Applying conditional identity:

$$
\exists x \neg(\neg P(x) \vee M(x))
$$

Applying De Morgan's law:

$$
\exists x(\neg \neg P(x) \wedge \neg M(x))
$$

Applying double negation law:

$$
\exists x(P(x) \wedge \neg M(x))
$$

English: There was a patient who was given a placebo and did not have migraines.


\section*{section e}
Logical Expression:

$$
\exists x(M(x) \wedge P(x))
$$

Negation:

$$
\neg \exists x(M(x) \wedge P(x))
$$

Applying De Morgan's law:

$$
\forall x \neg(M(x) \wedge P(x))
$$

Applying De Morgan's law:

$$
\forall x(\neg M(x) \vee \neg P(x))
$$

English: Every patient either did not have migraines or was not given a placebo.

\section*{2. Exercise 1.9.4}
\section*{section c}

$$
\begin{aligned}
\neg \exists x \forall y(P(x, y) \rightarrow Q(x, y)) & \equiv \forall x \neg \forall y(P(x, y) \rightarrow Q(x, y)& \text {(De Morgan's law)}\\
& \equiv \forall x \exists y \neg(P(x, y) \rightarrow Q(x, y)                                                      & \text {(De Morgan's law)}\\
& \equiv \forall x \exists y \neg(\neg P(x, y) \vee Q(x, y)                                                        & \text {(Conditional identity)}\\
& \equiv \forall x \exists y(\neg \neg P(x, y) \wedge \neg Q(x, y)                                                 & \text {(De Morgan's law)}\\
& \equiv \forall x \exists y(P(x, y) \wedge \neg Q(x, y))                                                          & \text {(Double negation law)}\\
\end{aligned}
$$

\section*{section d}

$$
\begin{array}{rlr}
\neg \exists x \forall y(P(x, y) \leftrightarrow P(y, x)) & \equiv \forall x \neg \forall y(P(x, y) \leftrightarrow P(y, x)) & \text { (De Morgan's law) } \\
& \equiv \forall x \exists y \neg(P(x, y) \leftrightarrow P(y, x)) & \text { (De Morgan's law) } \\
& \equiv \forall x \exists y \neg((P(x, y) \rightarrow P(y, x)) \wedge(P(y, x) \rightarrow P(x, y))) & \text { (Conditional identity) } \\
& \equiv \forall x \exists y(\neg(P(x, y) \rightarrow P(y, x)) \vee \neg(P(y, x) \rightarrow P(x, y))) & \text { (De Morgan's law) } \\
& \equiv \forall x \exists y(\neg(\neg P(x, y) \vee P(y, x)) \vee \neg(\neg P(y, x) \vee P(x, y))) & \text { (Conditional identity) } \\
& \equiv \forall x \exists y((\neg \neg P(x, y) \wedge \neg P(y, x)) \vee(\neg \neg P(y, x) \wedge \neg P(x, y))) & \text { (De Morgan's law) } \\
& \equiv \forall x \exists y((P(x, y) \wedge \neg P(y, x)) \vee(P(y, x) \wedge \neg P(x, y))) & \text { (Double negation law) }
\end{array}
$$

\section*{section e}

$$
\begin{aligned}
\neg(\exists x \exists y P(x, y) \wedge \forall x \forall y Q(x, y)) & \equiv \neg \exists x \exists y P(x, y) \vee \neg \forall x \forall y Q(x, y) & & \text { (De Morgan's law) } \\
& \equiv \forall x \neg \exists y P(x, y) \vee \exists x \neg \forall y Q(x, y) & & \text { (De Morgan's law) } \\
& \equiv \forall x \forall y \neg P(x, y) \vee \exists x \exists y \neg Q(x, y) & & \text { (De Morgan's law) }
\end{aligned}
$$
\end{document}