\documentclass[10pt]{article}
\usepackage[utf8]{inputenc}
\usepackage[T1]{fontenc}
\usepackage{amsmath}
\usepackage{amsfonts}
\usepackage{amssymb}
\usepackage[version=4]{mhchem}
\usepackage{stmaryrd}
\usepackage{array,mathtools}
\newcommand*{\carry}[1][1]{\overset{#1}}
\newcolumntype{B}[1]{r*{#1}{@{\,}r}}

\begin{document}
\section*{Homework 2}
Student: Kevin Fang (kf1325)

\section*{Question 5}
\section*{(a)}
\section*{1. Exercise 1.12.2}
\section*{section b}
\[
\begin{array}{rrr}
(1) & p \rightarrow (q \wedge r) & \text{Hypothesis} \\
(2) & \neg q & \text{Hypothesis} \\
(3) & \neg q \vee \neg r & \text{Addition, 2} \\
(4) & \neg(q \wedge r) & \text{De Morgan's law, 3} \\
(5) & \neg p & \text{Modus tollens, 4} \\
\end{array}
\]

\section*{section e}

\[
\begin{array}{rrr}
(1) & p \vee q & \text{Hypothesis} \\
(2) & \neg p \vee r & \text{Hypothesis} \\
(3) & \neg q & \text{Hypothesis} \\
(4) & p & \text{Disjunctive syllogism, 1, 3} \\
(5) & \neg \neg p & \text{Double negation law, 4} \\
(6) & r & \text{Disjunctive syllogism, 2, 5} \\
\end{array}
\]

\section*{2. Exercise 1.12.3}
\section*{section c}
\[
\begin{array}{rrr}
(1) & p \vee q & \text{Hypothesis}\\
(2) & \neg p & \text{Hypothesis}\\
(3) & \neg \neg p \vee q & \text{Double negation law, 1}\\
(4) & \neg p \rightarrow q & \text{Conditional identity, 3}\\ 
(5) & q & \text{Modus ponens, 2, 4}\\
\end{array}
\]

\section*{3. Exercise 1.12.5}
\section*{section c}
$p=\mathrm{I}$ will buy a new car\\
$q=$ I will buy a new house\\
$r=\mathrm{I}$ will get a job\\

\text{This argument can be written as following:} \\

$$
\begin{aligned}
& (p \wedge q) \rightarrow r \\
& \neg r \\
\hline
& \therefore \neg p
\end{aligned}
$$

\text{Thus, this argument is invalid.} \\

\text{Proof:} \\

{When $p$ is true, $q$ is false, and $r$ is false, both premises $(p \wedge q) \rightarrow r$ and $\neg r$ are true and the conclusion $\neg p$ is false. Since the premises are true and the conclusion is false, the argument is invalid.} \\

\section*{section d}
$p=$ I will buy a new car\\
$q=\mathrm{I}$ will buy a new house\\
$r=\mathrm{I}$ will get a job\\

\text{This argument can be written as following:}

$$
\begin{aligned}
& (p \wedge q) \rightarrow r \\
& \neg r \\
& q \\
\hline
& \therefore \neg p
\end{aligned}
$$

\text{Thus, this argument is invalid.} \\

\text{Proof:} \\

\[
\begin{array}{rrr}
(1) & (p \wedge q) \rightarrow r & \text{Hypothesis}\\
(2) & \neg r                     & \text{Hypothesis}\\
(3) & q                          & \text{Hypothesis}\\
(4) & \neg(p \wedge q)           & \text{Modus tollens, 1, 2}\\
(5) & \neg p \vee \neg q         & \text{De Morgan's law, 4}\\
(6) & \neg \neg q                & \text{Double negation law, 3}\\
(7) & \neg p                     & \text{Disjunctive syllogism, 5, 6}\\
\end{array}
\]

\section*{(b)}
\section*{1. Exercise 1.13.3}
\section*{section b}
\begin{center}
\begin{tabular}{|c|c|c|}
\hline
 & $P$ & $Q$ \\
\hline
$a$ & F & F \\
\hline
$b$ & F & T \\
\hline
\end{tabular}
\end{center}

{Both hypotheses $\exists x(P(x) \vee Q(x))$ and $\exists x \neg Q(x)$ are true for the values of $P$ and $Q$ on elements $a$ and $b$ given in the table. However, the conclusion $\exists x P(x)$ is false. Therefore, the argument is invalid.} \\

\section*{2. Exercise 1.13.5}
\section*{section d}

\text{The following predicates can be defined as followed:}\\

{$M(x): x$ missed class.}

{$D(x): x$ got a detention.}

\text{These arguments can be formed as followed:}\\

$$
\begin{aligned}
& \forall x(M(x) \rightarrow D(x))\\
& \text{Penelope is an element} \\
& \neg M (Penelope) \\
\hline
& \therefore \neg D (Penelope) \\
\end{aligned}
$$

\text{Thus, this argument is invalid.} \\

\text{Proof is shown in the following truth table:} \\

\begin{center}
\begin{tabular}{|c|c|c|}
\hline
 & $M$ & $D$ \\
\hline
Penelope & F & T \\
\hline
$b$ & F & T \\
\hline
\end{tabular}
\end{center}

All hypotheses, $\forall x(M(x) \rightarrow D(x))$, Penelope is a particular element, $\neg M$ (Penelope) are true for the values of $M$ and $D$ on elements $Penelope$ and $b$ given in the table. However, the conclusion $\neg D($ Penelope $)$ is false. Therefore, the argument is invalid.

\section*{section e}

\text{The following predicates can be defined as followed:}\\

{$M(x): x$ missed class.}

{$D(x): x$ got a detention.}

{$A(x): x$ got an A.}

\text{These arguments can be formed as followed:}\\

$$
\begin{aligned}
& \forall x((M(x) \vee D(x)) \rightarrow \neg A(x)) \\
& \text { Penelope is an element } \\
& \text { A(Penelope }) \\
\hline
& \therefore \neg D(\text { Penelope })
\end{aligned}
$$

\text{Thus, this argument is invalid.} \\

\text{Proof is shown as followed:} \\

\[
\begin{array}{rrr}
(1) & \forall x((M(x) \vee D(x)) \rightarrow \neg A(x))                                              & \text{Hypothesis}\\
(2) & \text { Penelope is an element }                                                               & \text{Hypothesis}\\
(3) & A(\text { Penelope })                                                                          & \text{Hypothesis}\\
(4) & (M(\text { Penelope }) \vee D(\text { Penelope })) \rightarrow \neg A(\text { Penelope })      & \text{Universal instantiation, 1, 2}\\
(5) & \neg \neg A(\text { Penelope })                                                                & \text{Double negation law, 3}\\
(6) & \neg(M(\text { Penelope }) \vee D(\text { Penelope }))                                         & \text{Modus tollens, 4,5}\\
(7) & \neg M(\text { Penelope }) \wedge \neg D(\text { Penelope })                                   & \text{De Morgan's law, 6}\\
(8) & \neg D(\text { Penelope })                                                                     & \text{Simplification, 7}\\
\end{array}
\]

\pagebreak
\section*{Question 6}
\section*{Exercise 2.4.1}
\section*{section d}
Proof:

Suppose $x$ and $y$ are two odd integers.\\
$x$ is an odd integer, $x=2 k+1$, for some integer $k$.\\
$y$ is an odd integer, $y=2 n+1$, for some integer $n$.\\
Substitute $x=2k+1$ and $y=2n+1$ into $x y$:

$$
\begin{aligned}
x y & =(2 k+1)(2 n+1) \\
& =4 k n+2 k+2 n+1 \\
& =2(2 k n+k+n)+1 \\
\end{aligned}
$$

$2 k n+k+n$ is an integer since $n$ and $k$ are integers.\\
Suppose integer $m=2 k n+k+n$, we have $x y$ in the form of $2 m+1$; thus, $x y$ is odd.\\
Therefore, the product of two odd integers is an odd integer.

\section*{Exercise 2.4.3}
\section*{section b}

Suppose $x$ is a real number and $x \leq 3$, we prove $12-7 x+x^{2} \geq 0$.\\

$$
\begin{array}{rr}
(1) & x \leq 3 \\
(2) & x-3 \leq 0 \\
(3) & x-4 \leq-1 \\
(4) & (x-3)(x-4) \geq 0 \\
\end{array}
$$

Since both (2) and (3) are less than or equal to 0, (4) is greater than or equal to 0. Distributing (4), we get the following form:
$$
\begin{array}{rr}
(5) & x^{2}-7 x+12 \geq 0 \\
\end{array}
$$

\pagebreak
\section*{Question 7}
\section*{Exercise 2.5.1}
\section*{section d}
Suppose $n$ is even, we prove $n^{2}-2 n+7$ is odd.\\
For any even integer $n$, $n$ follows the form $n=2 k$ for some integer $k$.\\

$$
\begin{aligned}
n^{2}-2 n+7 & =(2 k)^{2}-2(2 k)+7 \\
& =4 k^{2}-4 k+7 \\
& =2\left(2 k^{2}-2 k+3\right)+1
\end{aligned}
$$

The final result follows the form $2 m+1$ where the integer $m=2 k^{2}-2 k+3$. $n^{2}-2 n+7$ is odd.

\section*{Exercise 2.5.4}
\section*{section a}
Suppose $x>y$, we prove $x^{3}+x y^{2}>x^{2} y+y^{3}$, for every pair of real number $x$ and $y$.\\
Since $x$ and $y$ are real numbers,

$$
\begin{aligned}
x^{2} & \geq 0 \\
y^{2} & \geq 0
\end{aligned}
$$

Thus,

$$
x^{2}+y^{2} \geq 0
$$

If $x^{2}+y^{2}=0$, we have $x=0, y=0$; furthermore, $x=y$. Since $x>y$, $x^{2}+y^{2} \neq 0$
Thus,

$$
x^{2}+y^{2}>0
$$

Since $x>y$ and $x^{2}+y^{2}>0$

$$
\begin{array}{r}
x\left(x^{2}+y^{2}\right)>y\left(x^{2}+y^{2}\right) \\
x^{3}+x y^{2}>x^{2} y+y^{3}
\end{array}
$$

\section*{section b}
Suppose $x \leq 10$ and $y \leq 10$, we prove $x+y \leq 20$, for every pair of real number $x$ and $y$.

$$
\begin{array}{r}
x+y \leq 10+10 \\
x+y \leq 20
\end{array}
$$

\section*{Exercise 2.5.5}
\section*{section c}
Suppose $\frac{1}{r}$ is rational, we prove $r$ is rational.\\
If $\frac{1}{r}$ is rational, then $\frac{1}{r}=\frac{a}{b}$ for some integers $a$ and $b$, where $b \neq 0$.\\
We have:

$$
b=a r
$$

If $a=0$, then $a r=0$. \\
Since $a r=b$ and $b \neq 0$, we have $a \neq 0$.\\
Since $b=a r$ and $a \neq 0$, we have

$$
r=\frac{b}{a}
$$

Since both $a$ and $b$ are integers, $r$ must be rational.

\pagebreak
\section*{Question 8}
\section*{Exercise 2.6.6}
\section*{section c}
We disprove the statement that "the average of three real numbers is greater than any of these three numbers".

$x, y$ and $z$ are real numbers:

$$
\begin{aligned}
& \frac{x+y+z}{3}<x \\
& \frac{x+y+z}{3}<y \\
& \frac{x+y+z}{3}<z
\end{aligned}
$$

The sum of the three inequalities is as followed:

$$
x+y+z<x+y+z
$$

Since the above inequality is invalid, the statement that the average of three real numbers is greater than or equal to at least one of the numbers must be true.

\section*{section d}
We disprove the argument that "there is no smallest integer".\\
For a smallest integer $x$:

$$
x-1<x
$$

Since $x-1$ is less than $x$, there must be an integer that is smaller than $x$; thus, $x$ is not the smallest integer. This statement leads to a contradiction. Therefore, the statement that there is no smallest integer must be true.

\pagebreak
\section*{Question 9}
\section*{Exercise 2.7.2}
\section*{section b}
Proof:

Consider the following two cases:\\
Case 1: $x$ and $y$ are even, then $x=2 k$ and $y=2 j$, while $k$ and $j$ are both some integers.

$$
\begin{aligned}
x+y & =2 k+2 j \\
& =2(k+j)
\end{aligned}
$$

Since $x+y$ follows the form of $2 m$ for the integer $m=k+j$, $x+y$ must be even.\\
Case 2: $x$ and $y$ are odd, then $x=2 k+1$ and $y=2 j+1$, while $k$ and $j$ are both some integers.

$$
\begin{aligned}
x+y & =2 k+1+2 j+1 \\
& =2(k+j+1)
\end{aligned}
$$

Since $x+y$ follows the form of $2 m$ for the integer $m=k+j+1$, $x+y$ must be even.\\
In both cases, $x+y$ is even. Therefore, it holds true that "if integers $x$ and $y$ have the same parity, then $x+y$ is even".


\end{document}